\documentclass[handout]{beamer} % use [handout] option to stop pauses
\beamertemplatenavigationsymbolsempty 

\usepackage[utf8]{inputenc}
\usepackage{svg}
\usepackage{amsmath}
\usepackage{amssymb}
\usepackage{mathtools}
\usepackage{stmaryrd}
\usepackage{libertine}
\usepackage{xparse}
\usepackage{mathpartir}
\usepackage{fancybox}
\usepackage{graphicx}
\usepackage{tikz}
\usepackage{../jmsdelim}
\usepackage{../macros}
\usepackage{../categories}

\usecolortheme[named=BristolURed]{structure}

\AtBeginSection[]{
  \begin{frame}
  \vfill
  \centering
  \begin{beamercolorbox}[sep=12pt,center,shadow=true]{title}
		\usebeamerfont{title}
		\scshape
		\uppercase\expandafter{\romannumeral\insertsectionnumber\relax}.~\insertsectionhead\par%
  \end{beamercolorbox}
  \vfill
  \end{frame}
}

\newcommand\doubleplus{+\kern-1.3ex+\kern0.8ex}

\title{Type Theory and Homotopy \\ III. Universes, Univalence, and Quotients}
\author{
		Alex Kavvos % \inst{1} % \orcidID{0000-0001-7953-7975}
}

% \institute{
% 		% \inst{1}
%     University of Bristol
% 		% \email{alex.kavvos@bristol.ac.uk}
% }
\titlegraphic{\includegraphics[scale=0.1]{../Bristol.png}}
\date{Panhellenic Logic Symposium, 6--10 July 2022}


\begin{document}

\frame{\titlepage}

\begin{frame}
  \frametitle{Previously}

  Intensional identity types seem to support the following view:
  \begin{itemize}
    \item Types are \textbf{spaces} (up to homotopy).
    \item Terms are \textbf{points}.
    \item Elements of the identity type are \textbf{paths}.
    \item Everything given in a \textbf{synthetic} manner, not analytic.
  \end{itemize}
  
  This discovery is independently due to 
  \begin{itemize}
    \item Awodey and Warren [Math. Proc. Camb. Philos. Soc. 2009]
    \item Vladimir Voevodsky (1966--2017) [Stanford lecture 2006]
  \end{itemize}
  
  Interpretation of TT into \textbf{simplicial sets}: Kapulkin and Lumsdaine,
  with thanks to Voevodsky [J. Eur. Math. Soc. 2018].
  
  \medskip
  
  But Voevodsky discovered that simplicial sets support a little bit more than
  MLTT\ldots
\end{frame}

\section{Universes}
\begin{frame}
  \frametitle{Identity types are not good enough}
  
  \begin{theorem}[Jan Smith, J. Symb. Log. 1988]
    The type correspondong to Peano's fourth axiom, i.e.
    \[
      \IsTy[n : \Nat]{\Id{\Nat}{0}{\Succ{n}} \to \mathbf{0}}[]
    \]
    is \textbf{not} inhabited in MLTT with $\Pi$, $\Sigma$, and identity types.
  \end{theorem}
  Proof: construct a model of MLTT where types are subsingleton sets.

  \medskip

  To prove Peano 4, we intuitively want to
  \begin{enumerate}
    \item \Alert{construct a type family} $\IsTy[n : \Nat]{B(n)}[]$ for which
      $B(\Tm{\Zero})$ is inhabited, while $\IsTy[n : \Nat]{B(\Tm{\Succ{n}})}[]$
      is always empty.
    \item assuming $\IsTm[n : \Nat]{p}{\Id{\Nat}{0}{\Succ{n}}}$ and
      $\IsTm[]{M}{B(\Tm{\Zero})}$, obtain $\IsTm[n :
      \Nat]{\App{\Transport{p}}{M}}{B(\Tm{\Succ{n}})} \deq \textbf{0}$
  \end{enumerate}
  We cannot perform Step 1 because types are not terms.
\end{frame}

\begin{frame}
  \frametitle{Universes \`{a} la Russell}
  We introduce the \textbf{universe}, a \textbf{type of all (small) types}.
  \begin{mathpar}
    \inferrule{ }{\IsTy{\Uni}[]}
    \and
    \inferrule{
      \IsTm{A}{\Uni}
    }{
      \IsTy{A}[]
    }
  \end{mathpar}
  plus one rule for each type constructor, e.g.
  \begin{mathpar}
    \inferrule{
      \IsTm{A}{\Uni} \\
      \IsTm[\Gamma, x : \Uni]{B}{\Uni}
    }{
      \IsTm{\Fn{x}{A}{B}}{\Uni}
    }
  \end{mathpar}
  \Alert{Caution}. We must \textbf{not} have
  \[
    \inferrule{ }{\IsTm{\Uni}{\Uni}}
  \]
  because that causes a Russell-style paradox (Girard's paradox).
  
  \medskip
  
  Types may then be constructed as terms of $\Uni$ (e.g. by induction).
\end{frame}

\section{Homotopy Equivalences}

\begin{frame}
  \frametitle{Homotopy equivalence}

  \begin{definition}
    Two topological spaces $X$ and $Y$ are \textbf{homotopy-equivalent} if there
    are continuous functions $f : X \to Y$ and $g : Y \to X$ such that
    \begin{align*}
      g \circ f &\sim 1_X 
      &
      f \circ g &\sim 1_Y
    \end{align*}
    where $1_X$ and $1_Y$ are the identity functions on $X$ and $Y$.
  \end{definition}

  \begin{center}
    \includegraphics[scale=0.3]{mug.jpg}
  \end{center}
  
  We can model this synthetically in MLTT.
\end{frame}

\begin{frame}
  \frametitle{Type-theoretic homotopies}

  Let $A, B : \Uni$, and $f, g : A \to B$. 
  
  \medskip
  
  Define the type of \textbf{(type-theoretic) homotopies} from $f$ to $g$ to be
  \[
    f \sim g \defeq
      \Fn{x}{A}{
        \Id{B}{\App{f}{x}}{\App{g}{x}}
      }
  \]
  
  Given $f : A \to B$ and $g : B \to C$ define 
  $
    g \circ f \defeq
      \Lam{x}{\App{g}{\App{f}{x}}} : A \to C
  $.
  
  \medskip
  
  Given $f : A \to B$ define the type of its \textbf{quasi-inverses} to be
  \[
    \QInv{f} \defeq
      \DSum{g}{B \to A}{
        (g \circ f \sim 1_A) \times (f \circ g \sim 1_B)
      }
  \]
  where $1_A : A \to A$ is given by $1_A \defeq \Lam{x}[A]{x}$.

  \medskip
  
  There is a type $\IsEquiv{f}$ with
  \begin{align*}
    \QInv{f} &\to \IsEquiv{f}
    &
    \IsEquiv{f} &\to \QInv{f}
  \end{align*}
  which is significantly better behaved than $\QInv{f}$.
\end{frame}

\begin{frame}
  \frametitle{Type-theoretic equivalences and univalence}
  
  For $A, B : \Uni$ define the type of \textbf{(type-theoretic) equivalences}
  \[
    A \simeq B \defeq
      \DSum{f}{A \to B}{\IsEquiv{f}}
  \]
  
  We can use equivalences to decompose identity types.

  For example, for any $A, B : \Uni$ and $p, q : A \times B$:
  \begin{align*}
    \Id{A \times B}{p}{q} &\simeq 
    \Id{A}{\Proj[1]{p}}{\Proj[1]{q}} \times \Id{B}{\Proj[2]{p}}{\Proj[2]{q}}
  \end{align*}
  
  \textbf{Question}: what is an identity between types?
  
  \medskip

  Voevodsky proposed adding the \textbf{univalence axiom}:
  \[
    \textsf{ua} : 
      \Fn{A, B}{\Uni}{
        (A \simeq B) \simeq \Id{\Uni}{A}{B}
      }
  \]
\end{frame}

\end{document}