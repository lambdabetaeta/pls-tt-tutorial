\documentclass[handout]{beamer} % use [handout] option to stop pauses
\beamertemplatenavigationsymbolsempty 

\usepackage[utf8]{inputenc}
\definecolor{BristolURed}{HTML}{B01C2E} % UBC Blue (primary)

\usecolortheme[named=BristolURed]{structure}

\usepackage{amsmath}
\usepackage{amssymb}
\usepackage{mathtools}
\usepackage{stmaryrd}
\usepackage{libertine}
\usepackage{xparse}
\usepackage{mathpartir}
\usepackage{fancybox}
\usepackage{jmsdelim}
\usepackage{macros}

\AtBeginSection[]{
  \begin{frame}
  \vfill
  \centering
  \begin{beamercolorbox}[sep=12pt,center,shadow=true]{title}
		\usebeamerfont{title}
		\scshape
		\uppercase\expandafter{\romannumeral\insertsectionnumber\relax}.~\insertsectionhead\par%
  \end{beamercolorbox}
  \vfill
  \end{frame}
}

\newcommand\doubleplus{+\kern-1.3ex+\kern0.8ex}

\title{Type Theory and Homotopy \\ I. Constructions and dependence}
\author{
		Alex Kavvos % \inst{1} % \orcidID{0000-0001-7953-7975}
}

\institute{
		% \inst{1}
    University of Bristol
		% \email{alex.kavvos@bristol.ac.uk}
}
\date{Panhellenic Logic Symposium, 6--10 July 2022}


\begin{document}

\frame{\titlepage}


\section{Intuitionism and Constructions}

\begin{frame}
  \frametitle{Intuitionism, Constructivism, and Type Theory}
  
  Some general talk.
  
\end{frame}

\begin{frame}
\frametitle{Constructions}

Let $A, B, \dots$ be sets.
\begin{align*}
  \mathbf{0} &\defeq \emptyset &
  \mathbf{1} &\defeq \{ \ast \}  \\
  A \times B &\defeq \{ (a, b) \mid a \in A \text{ and } b \in B \} &
  A \to B &\defeq \{ f \mid f : A \to B \}
\end{align*}
\[
  A + B \defeq \{ (1, a) \mid a \in A \} \cup \{ (2, y) \mid b \in B \}
\]
Let $\lnot A \defeq A \to \mathbf{0}$.

\begin{example}
  \small
  \begin{itemize}
    \item $(x, y) \mapsto x \in (A \times B) \to A$
    \item $x \mapsto (y \mapsto (x, y)) \in A \to (B \to A \times B)$
    \item $\lambda x.\, \lambda y.\, (x, y) \in A \to (B \to A \times B)$
    \item $\lambda (x, v). \begin{cases}
        (1, (x, a)) & \text{ if } v = (1, a) \\
        (2, (x, b)) & \text{ if } v = (2, b) \\
    \end{cases}
    \in A \times (B + C) \to (A \times B) + (A \times C)$
    \item $\lambda a.\, \lambda f.\, f(a) \in A \to \lnot\lnot A$
  \end{itemize}
\end{example}

\end{frame}

\begin{frame}
  \frametitle{Dependence}
  Let $(B_a)_{a \in A}$ be a \textbf{family} of sets.

  \begin{align*}
    \Prod{a}{A}{B} &\defeq \sum_{a \in A} B_a \defeq
      \left\{ (a, b) \mid a \in A \text{ and } b \in B_a \right\}  \\
    \Fn{a}{A}{B} &\defeq \prod_{a \in A} B_a \defeq
      \left\{ f : A \to \bigcup_{a \in A} B_a \mid f(a) \in B_a \text{ for all } a \in A \right\}  \\
  \end{align*}

  Given a \textbf{constant} family of sets $(B)_{a \in A}$ we have
  \begin{align*}
    \Prod{a}{A}{B} &= A \times B &
    \Fn{a}{A}{B} &= A \to B
  \end{align*}
  
  \begin{example}
    Let $P_n \defeq \begin{cases}
      \{ \ast \} & \text{ if } $n$ \text{ is prime} \\
      \emptyset & \text{ otherwise }
    \end{cases}$
    \begin{itemize}
      \item $(11, *) \in \Prod{n}{\mathbb{N}}{P_n}$, but 
      $(4, *) \not\in \Prod{n}{\mathbb{N}}{P_n}$ \\
      \item $\lambda n.\, \textsf{if } n \text{ is prime} \textsf{ then } (1, *) \textsf{ else } (2, \textsf{id}_{\emptyset}) \in \Fn{n}{\mathbb{N}}{P_n + \lnot P_n}$
    \end{itemize}
  \end{example}
\end{frame}



\section{Martin-L\"of Type Theory}

\begin{frame}
  \frametitle{Judgements}
  
  Martin-L\"of Type Theory (MLTT) is a \textbf{natural deduction} system.

  It has six distinct kinds of \textbf{judgement}:
  \begin{center}
    \begin{tabular}{ll}
      $\IsCtx{\Gamma}$    & $\Gamma$ is a context \\
      $\IsTy{A}[]$        & $A$ is a type in context $\Gamma$ \\
      $\IsTm{M}{A}$       & $M$ is a term of type $A$ in context $\Gamma$ \\
      & \\
      $\EqCtx{\Gamma}{\Delta}$  & $\Gamma$ and $\Delta$ are equal contexts \\
      $\EqTy{A}{B}[]$             & $A$ and $B$ are equal types in context $\Gamma$ \\
      $\EqTm{M}{N}{A}$          & $M$ and $N$ are equal terms of type $A$ in context $\Gamma$
    \end{tabular}
  \end{center}
  These are generated \textbf{inductive-inductively}.

  The equality judgements have rules that make them
  \begin{itemize}
    \item equivalence relations, e.g. $
        \inferrule{
          \EqTy{A}{B}[]
        }{
          \EqTy{B}{A}[]
        }
    $
    \item congruences 
  \end{itemize}
\end{frame}

\begin{frame}
  \frametitle{What is a type?}
  
  \begin{block}{Ingredients of a type}
    \begin{itemize}
      \item a \textbf{formation} rule (when can I form this type?)
      \item an \textbf{introduction} rule (how do I make elem's of this type?)
      \item an \textbf{elimination} rule (how do I use elem's of this type?)
      \item a \textbf{computation} rule (how do I calculate with its elements?)
      \item a \textbf{uniqueness} rule
    \end{itemize}
  \end{block}
\end{frame}

\begin{frame}
  \frametitle{Dependent product types / $\Pi$ types}
  
  \begin{center}
      \renewcommand{\arraystretch}{2.5}
    \begin{tabular}{p{2cm}c}
      formation &
      $
        \inferrule{
          \IsTy[\Gamma, x : A]{B}[]
        }{
          \IsTy{\Fn{x}{A}{B}}[]
        }
      $ \\
      introduction &
      $
        \inferrule{
          \IsTm[\Gamma, x : A]{M}{B}
        }{
          \IsTm{\Lam{x}{M}}{\Fn{x}{A}{B}}
        }
      $ \\
      elimination &
      $
        \inferrule{
          \IsTm{M}{\Fn{x}{A}{B}} \\
          \IsTm{N}{A}
        }{
          \IsTm{\App{M}{N}}{\Sb{B}{N}{x}}
        }
      $ \\
      computation &
      $
        \inferrule{
          \IsTm[\Gamma, x : A]{M}{B} \\
          \IsTm{N}{A}
        }{
          \EqTm{\App*{\Lam{x}{M}}{N}}{\Sb{M}{N}{x}}{\Sb{B}{N}{x}}
        }
      $ \\
      uniqueness &
      $
        \inferrule{
          \IsTm{M}{\Fn{x}{A}{B}}
        }{
          \EqTm{M}{\Lam{x}{\App{M}{x}}}{\Fn{x}{A}{B}}
        }
      $
    \end{tabular}
  \end{center}
\end{frame}

\begin{frame}
  \frametitle{Dependent sum types / $\Sigma$ types}
  
  \begin{center}
      \renewcommand{\arraystretch}{2.5}
    \begin{tabular}{p{2cm}c}
      formation &
      $
        \inferrule{
          \IsTy[\Gamma, x : A]{B}[]
        }{
          \IsTy{\Prod{x}{A}{B}}[]
        }
      $ \\
      introduction &
      $
        \inferrule{
          \IsTy[\Gamma, x : A]{B}[] \\
          \IsTm{M}{A} \\
          \IsTm{N}{\Sb{B}{M}{x}}
        }{
          \IsTm{\Pair{M}{N}}{\Prod{x}{A}{B}}
        }
      $ \\
      elimination &
      $
        \inferrule{
          \IsTm{M}{\Prod{x}{A}{B}} \\
        }{
          \IsTm{\Proj[1]{M}}{A}
        }
      $ 
      $
        \inferrule{
          \IsTm{M}{\Prod{x}{A}{B}} \\
        }{
          \IsTm{\Proj[2]{M}}{\Sb{B}{\Proj[1]{M}}{x}}
        }
      $ 
      \\
      computation &
      $
        \inferrule{
          \IsTy[\Gamma, x : A]{B}[] \\
          \IsTm{M}{A} \\
          \IsTm{N}{\Sb{B}{M}{x}}
        }{
          \EqTm{\Proj[1]{\Pair{M}{N}}}{M}{A}
        }
      $ \\
      &
      $
        \inferrule{
          \IsTy[\Gamma, x : A]{B}[] \\
          \IsTm{M}{A} \\
          \IsTm{N}{\Sb{B}{M}{x}}
        }{
          \EqTm{\Proj[2]{\Pair{M}{N}}}{N}{\Sb{B}{\Proj[1]{M}}{x}}
        }
      $
      \\
      uniqueness &
      $
        \inferrule{
          \IsTm{M}{\Prod{x}{A}{B}}
        }{
          \EqTm{M}{\Pair{\Proj[1]{M}}{\Proj[2]{M}}}{\Prod{x}{A}{B}}
        }
      $
    \end{tabular}
  \end{center}
\end{frame}

\begin{frame}
  \frametitle{Natural numbers}
  
  \begin{center}
      \renewcommand{\arraystretch}{2.5}
    \begin{tabular}{p{2cm}c}
      formation &
      $
        \inferrule{
          \IsTy[\Gamma, x : A]{B}[]
        }{
          \IsTy{\Prod{x}{A}{B}}[]
        }
      $ \\
      introduction &
      $
        \inferrule{
          \IsTy[\Gamma, x : A]{B}[] \\
          \IsTm{M}{A} \\
          \IsTm{N}{\Sb{B}{M}{x}}
        }{
          \IsTm{\Pair{M}{N}}{\Prod{x}{A}{B}}
        }
      $ \\
      elimination &
      $
        \inferrule{
          \IsTm{M}{\Prod{x}{A}{B}} \\
        }{
          \IsTm{\Proj[1]{M}}{A}
        }
      $ 
      $
        \inferrule{
          \IsTm{M}{\Prod{x}{A}{B}} \\
        }{
          \IsTm{\Proj[2]{M}}{\Sb{B}{\Proj[1]{M}}{x}}
        }
      $ 
      \\
      computation &
      $
        \inferrule{
          \IsTy[\Gamma, x : A]{B}[] \\
          \IsTm{M}{A} \\
          \IsTm{N}{\Sb{B}{M}{x}}
        }{
          \EqTm{\Proj[1]{\Pair{M}{N}}}{M}{A}
        }
      $ \\
      &
      $
        \inferrule{
          \IsTy[\Gamma, x : A]{B}[] \\
          \IsTm{M}{A} \\
          \IsTm{N}{\Sb{B}{M}{x}}
        }{
          \EqTm{\Proj[2]{\Pair{M}{N}}}{N}{\Sb{B}{\Proj[1]{M}}{x}}
        }
      $
      \\
      uniqueness &
      $
        \inferrule{
          \IsTm{M}{\Prod{x}{A}{B}}
        }{
          \EqTm{M}{\Pair{\Proj[1]{M}}{\Proj[2]{M}}}{\Prod{x}{A}{B}}
        }
      $
    \end{tabular}
  \end{center}
\end{frame}


\section{Examples}



\section{Universes}


\end{document}