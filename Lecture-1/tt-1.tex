\documentclass[handout]{beamer} % use [handout] option to stop pauses
\beamertemplatenavigationsymbolsempty 

\usepackage[utf8]{inputenc}
\definecolor{BristolURed}{HTML}{B01C2E} % UBC Blue (primary)

\usecolortheme[named=BristolURed]{structure}

\usepackage{amsmath}
\usepackage{amssymb}
\usepackage{mathtools}
\usepackage{stmaryrd}
\usepackage{libertine}
\usepackage{xparse}
\usepackage{mathpartir}
\usepackage{fancybox}
\usepackage{jmsdelim}
\usepackage{macros}

\AtBeginSection[]{
  \begin{frame}
  \vfill
  \centering
  \begin{beamercolorbox}[sep=12pt,center,shadow=true]{title}
		\usebeamerfont{title}
		\scshape
		\uppercase\expandafter{\romannumeral\insertsectionnumber\relax}.~\insertsectionhead\par%
  \end{beamercolorbox}
  \vfill
  \end{frame}
}

\newcommand\doubleplus{+\kern-1.3ex+\kern0.8ex}

\title{Type Theory and Homotopy \\ I. Constructions and dependence}
\author{
		Alex Kavvos % \inst{1} % \orcidID{0000-0001-7953-7975}
}

\institute{
		% \inst{1}
    University of Bristol
		% \email{alex.kavvos@bristol.ac.uk}
}
\date{Panhellenic Logic Symposium, 6--10 July 2022}


\begin{document}

\frame{\titlepage}


\section{Intuitionism and Constructions}

\begin{frame}
  \frametitle{Intuitionism, Constructivism, and Type Theory}
  
  Some general talk.
  
\end{frame}

\begin{frame}
\frametitle{Constructions}

Let $A, B, \dots$ be sets.
\begin{align*}
  \mathbf{0} &\defeq \emptyset &
  \mathbf{1} &\defeq \{ \ast \}  \\
  A \times B &\defeq \{ (a, b) \mid a \in A \text{ and } b \in B \} &
  A \to B &\defeq \{ f \mid f : A \to B \}
\end{align*}
\[
  A + B \defeq \{ (1, a) \mid a \in A \} \cup \{ (2, y) \mid b \in B \}
\]
Let $\lnot A \defeq A \to \mathbf{0}$.

\begin{example}
  \small
  \begin{itemize}
    \item $(x, y) \mapsto x \in (A \times B) \to A$
    \item $x \mapsto (y \mapsto (x, y)) \in A \to (B \to A \times B)$
    \item $\lambda x.\, \lambda y.\, (x, y) \in A \to (B \to A \times B)$
    \item $\lambda (x, v). \begin{cases}
        (1, (x, a)) & \text{ if } v = (1, a) \\
        (2, (x, b)) & \text{ if } v = (2, b) \\
    \end{cases}
    \in A \times (B + C) \to (A \times B) + (A \times C)$
    \item $\lambda a.\, \lambda f.\, f(a) \in A \to \lnot\lnot A$
  \end{itemize}
\end{example}

\end{frame}

\begin{frame}
  \frametitle{Dependence}
  Let $(B_a)_{a \in A}$ be a \textbf{family} of sets.

  \begin{align*}
    \DSum{a}{A}{B_a} &\defeq \sum_{a \in A} B_a \defeq
      \left\{ (a, b) \mid a \in A \text{ and } b \in B_a \right\}  \\
    \Fn{a}{A}{B_a} &\defeq \prod_{a \in A} B_a \defeq
      \left\{ f : A \to \bigcup_{a \in A} B_a \mid f(a) \in B_a \text{ for all } a \in A \right\}  \\
  \end{align*}

  Given a \textbf{constant} family of sets $(B)_{a \in A}$ we have
  \begin{align*}
    \DSum{a}{A}{B} &= A \times B &
    \Fn{a}{A}{B} &= A \to B
  \end{align*}
  
  \begin{example}
    Let $P_n \defeq \begin{cases}
      \{ \ast \} & \text{ if } $n$ \text{ is prime} \\
      \emptyset & \text{ otherwise }
    \end{cases}$
    \begin{itemize}
      \item $(11, *) \in \DSum{n}{\mathbb{N}}{P_n}$, but 
      $(4, *) \not\in \DSum{n}{\mathbb{N}}{P_n}$ \\
      \item $\lambda n.\, \textsf{if } n \text{ is prime} \textsf{ then } (1, *) \textsf{ else } (2, \textsf{id}_{\emptyset}) \in \Fn{n}{\mathbb{N}}{P_n + \lnot P_n}$
    \end{itemize}
  \end{example}
\end{frame}



\section{Martin-L\"of Type Theory}

\begin{frame}
  \frametitle{Martin-L\"of Type Theory (MLTT)}

  \begin{itemize}
    \item Invented in the late 1960s to formalise constructive maths.
    \item A formal theory in \textbf{natural deduction} style.
    \item Every term in the theory needs to have a \textbf{type}.
    \item There are \textbf{no propositions}, only types.
  \end{itemize}
  
  \begin{center}
    \shadowbox{
      \begin{minipage}{0.30\textwidth}
        \centering
        propositions = types\\
        terms = proofs
      \end{minipage}
    }
  \end{center}
\end{frame}

\begin{frame}
  \frametitle{Judgements}
  
  Six distinct kinds of \textbf{judgement}:
  \begin{center}
    \begin{tabular}{ll}
      $\IsCtx{\Gamma}$    & $\Gamma$ is a context \\
      $\IsTy{A}[]$        & $A$ is a type in context $\Gamma$ \\
      $\IsTm{M}{A}$       & $M$ is a term of type $A$ in context $\Gamma$ \\
      & \\
      $\EqCtx{\Gamma}{\Delta}$  & $\Gamma$ and $\Delta$ are \textbf{definitionally} equal contexts \\
      $\EqTy{A}{B}[]$             & $A$ and $B$ are \textbf{definitionally} equal types \\
      $\EqTm{M}{N}{A}$          & $M$ and $N$ are \textbf{definitionally} equal terms
    \end{tabular}
  \end{center}
  The equality judgements have rules that make them
  \begin{itemize}
    \item equivalence relations, e.g. $
        \inferrule{
          \EqTy{A}{B}[]
        }{
          \EqTy{B}{A}[]
        }
    $
    \item congruences, e.g. \[
      \inferrule{
        \EqTy{A_1}{A_2}[] \\
        \EqTy[\Gamma, x : A_1]{B_1}{B_2}[]
      }{
        \EqTy{\Fn{x}{A_1}{B_1}}{\Fn{x}{A_2}{B_2}}[]
      }
    \]
  \end{itemize}
\end{frame}

\begin{frame}
  \frametitle{Contexts, variables, conversion}
  A \textbf{context} is a list of variables and their types.
  \begin{mathpar}
    \inferrule{ }{
      \IsCtx{\Emp}
    }
    \and
    \inferrule{
      \IsCtx{\Gamma} \\
      \IsTy{A}[]
    }{
      \IsCtx{\ECx{\Gamma}{x}{A}}
    }
  \end{mathpar}
  
  \medskip

  \textbf{Variables} stand for terms.

  If I have a variable I can use it as a term:
  \[
    \inferrule{
      \IsCtx{\Gamma, x : A, \Delta}
    }{
      \IsTm[\Gamma, x : A, \Delta]{x}{A}
    }
  \]
  
  \medskip
  
  We can always replace definitionally equals by equals. 

  The \textbf{type conversion} rule:
  \[
    \inferrule{
      \IsTm{M}{A} \\
      \EqTy{A}{B}[]
    }{
      \IsTm{M}{B}
    }
  \]

\end{frame}

\begin{frame}
  \frametitle{What is a type?}
  
  It is a classifier of terms.

  \medskip
  
  Terms of a certain type have an \textbf{interface}: a specification of how
  they can be created and consumed.
  
  \medskip
  
  \begin{block}{Ingredients of a type}
    \begin{itemize}
      \item a \textbf{formation} rule (when can I form this type?)
      \item an \textbf{introduction} rule (how do I make terms of this type?)
      \item an \textbf{elimination} rule (how do I use terms of this type?)
      \item a \textbf{computation} rule (how do I calculate with its elements?)
      \item a \textbf{uniqueness} rule (what do terms of this type look like?)
    \end{itemize}
  \end{block}
  Sometimes computation rules are called \textbf{$\beta$ rules}
  
  and uniqueness rules \textbf{$\eta$ rules}.
\end{frame}

\begin{frame}
  \frametitle{Dependent function types / $\Pi$ types}
  
  \begin{center}
      \renewcommand{\arraystretch}{2.5}
    \begin{tabular}{p{2cm}c}
      formation &
      $
        \inferrule{
          \IsTy[\Gamma, x : A]{B}[]
        }{
          \IsTy{\Fn{x}{A}{B}}[]
        }
      $ \\
      introduction &
      $
        \inferrule{
          \IsTm[\Gamma, x : A]{M}{B}
        }{
          \IsTm{\Lam{x}[A]{M}}{\Fn{x}{A}{B}}
        }
      $ \\
      elimination &
      $
        \inferrule{
          \IsTm{M}{\Fn{x}{A}{B}} \\
          \IsTm{N}{A}
        }{
          \IsTm{\App{M}{N}}{\Sb{B}{N}{x}}
        }
      $ \\
      computation &
      $
        \inferrule{
          \IsTm[\Gamma, x : A]{M}{B} \\
          \IsTm{N}{A}
        }{
          \EqTm{\App*{\Lam{x}[A]{M}}{N}}{\Sb{M}{N}{x}}{\Sb{B}{N}{x}}
        }
      $ \\
      uniqueness &
      $
        \inferrule{
          \IsTm{M}{\Fn{x}{A}{B}}
        }{
          \EqTm{M}{\Lam{x}[A]{\App{M}{x}}}{\Fn{x}{A}{B}}
        }
      $
    \end{tabular}
  \end{center}
\end{frame}

\begin{frame}
  \frametitle{Dependent sum types / $\Sigma$ types}
  
  \small
  
  \begin{center}
      \renewcommand{\arraystretch}{2.5}
    \begin{tabular}{p{2cm}c}
      formation &
      $
        \inferrule{
          \IsTy[\Gamma, x : A]{B}[]
        }{
          \IsTy{\DSum{x}{A}{B}}[]
        }
      $ \\
      introduction &
      $
        \inferrule{
          \IsTy[\Gamma, x : A]{B}[] \\
          \IsTm{M}{A} \\
          \IsTm{N}{\Sb{B}{M}{x}}
        }{
          \IsTm{\Pair{M}{N}}{\DSum{x}{A}{B}}
        }
      $ \\
      elimination &
      $
        \inferrule{
          \IsTm{M}{\DSum{x}{A}{B}} \\
        }{
          \IsTm{\Proj[1]{M}}{A}
        }
      $ 
      $
        \inferrule{
          \IsTm{M}{\DSum{x}{A}{B}} \\
        }{
          \IsTm{\Proj[2]{M}}{\Sb{B}{\Proj[1]{M}}{x}}
        }
      $ 
      \\
      computation &
      $
        \inferrule{
          \IsTy[\Gamma, x : A]{B}[] \\
          \IsTm{M}{A} \\
          \IsTm{N}{\Sb{B}{M}{x}}
        }{
          \EqTm{\Proj[1]{\Pair{M}{N}}}{M}{A}
        }
      $ \\
      &
      $
        \inferrule{
          \IsTy[\Gamma, x : A]{B}[] \\
          \IsTm{M}{A} \\
          \IsTm{N}{\Sb{B}{M}{x}}
        }{
          \EqTm{\Proj[2]{\Pair{M}{N}}}{N}{\Sb{B}{M}{x}}
        }
      $
      \\
      uniqueness &
      $
        \inferrule{
          \IsTm{M}{\DSum{x}{A}{B}}
        }{
          \EqTm{M}{\Pair{\Proj[1]{M}}{\Proj[2]{M}}}{\DSum{x}{A}{B}}
        }
      $
    \end{tabular}
  \end{center}
\end{frame}


\begin{frame}
  \frametitle{Coproducts (disjoint unions)}

  \begin{center}
      \renewcommand{\arraystretch}{2.5}
    \begin{tabular}{p{1cm}c}
      form. &
      $
        \inferrule{
          \IsTy{A}[] \\
          \IsTy{B}[]
        }{
          \IsTy{A + B}[]
        }
      $ \\
      intro. &
      $
        \inferrule{
          \IsTm{M}{A} \\
        }{
          \IsTm{\Inl{M}}{A + B}
        }
      $ 
      $
        \inferrule{
          \IsTm{N}{B} \\
        }{
          \IsTm{\Inr{N}}{A + B}
        }
      $ 
      \\[2ex]
      elim. &
      $
        \inferrule{
          \IsTm{M}{A + B} \\
          \IsTy[\Gamma, c : A + B]{C}[] \\
          \IsTm[\Gamma, x : A]{P}{\Sb{C}{\Inl{x}}{c}} \\
          \IsTm[\Gamma, y : B]{Q}{\Sb{C}{\Inr{y}}{c}}
        }{
          \IsTm{\Case{c. C}{M}{x}{P}{y}{Q}}{\Sb{C}{M}{c}}
        }
      $ \\[2ex]
      comp. &
      $
        \inferrule{
          \IsTm{M}{A + B} \\
          \IsTy[\Gamma, c : A + B]{C}[] \\
          \IsTm[\Gamma, x : A]{P}{\Sb{C}{\Inl{x}}{c}} \\
          \IsTm[\Gamma, y : B]{Q}{\Sb{C}{\Inr{y}}{c}} \\
          \IsTm{E}{A}
        }{
          \EqTm{\Case{c. C}{\Inl{E}}{x}{P}{y}{Q}}{\Sb{P}{E}{x}}{\Sb{C}{\Inl{E}}{c}}
        }
      $
    \end{tabular}
  \end{center}
\end{frame}

\begin{frame}
  \frametitle{Natural numbers}
  
  \small
  
  \begin{center}
    \renewcommand{\arraystretch}{2.5}
    \begin{tabular}{p{1.5cm}c}
      form. &
      $
        \inferrule{
          %
        }{
          \IsTy{\Nat}[]
        }
      $ \\
      intro. &
      $
        \inferrule{
          %
        }{
          \IsTm{\Zero}{\Nat}
        }
      $
      $
        \inferrule{
          \IsTm{N}{\Nat}
        }{
          \IsTm{\Succ{N}}{\Nat}
        }
      $
      \\[3ex]
      elim. &
      $
        \inferrule{
          \IsTm{N}{\Nat} \\
          \IsTy[\Gamma, n : \Nat]{C}[] \\
          \IsTm{P}{\Sb{C}{\Zero}{n}} \\
          \IsTm[\Gamma, n : \Nat, c : C]{Q}{\Sb{C}{\Succ{n}}{n}} \\
        }{
          \IsTm{\NatRec{c. C}{N}{P}{n}{c}{Q}}{\Sb{C}{N}{n}}
        }
      $
       \\
      comp. &
      $
        \inferrule{
          \ldots
        }{
          \EqTm{\NatRec{c. C}{\Zero}{P}{n}{c}{Q}}{P}{\Sb{C}{\Zero}{n}}
        }
      $
    \end{tabular}
      \[
        \inferrule{
          \ldots
        }{
          \EqTm{\NatRec{c. C}{\Succ{x}}{P}{n}{c}{Q}}{
            \Sb{Q}{x, \NatRec{c. C}{x}{P}{n}{c}{Q}}{n, c}
          }{\Sb{C}{\Succ{x}}{n}}
        }
      \]
  \end{center}
\end{frame}

\begin{frame}
  \frametitle{Metatheory}
  
  \begin{theorem}[Decidability of type checking]
    Given $\Gamma$, $\Tm{M}$, and $A$, it is decidable whether $\IsTm{M}{A}$.
  \end{theorem}
  
  \begin{theorem}[Canonicity]
    Let $\IsTm[]{M}{C}$. Then:
    \begin{itemize}
      \item if $C = A + B$ then either $\EqTm[]{M}{\Inl{P}}{A + B}$ for some $\IsTm[]{P}{A}$ or $\EqTm[]{N}{\Inr{Q}}{A + B}$ for some $\IsTm[]{Q}{B}$,
      \item if $C = \Nat$ then $\EqTm[]{M}{\textsf{succ}^n(\Zero)}{\Nat}$ for some $n \in \mathbb{N}$
      \item if $C = \DSum{x}{A}{B}$ then $\EqTm[]{M}{\Pair{P}{Q}}{\DSum{x}{A}{B}}$ for some $\IsTm[]{P}{A}$ and $\IsTm[]{Q}{\Sb{B}{P}{x}}$
    \end{itemize}
    Moreover, finding the ``canonical form'' of such terms is computable.
  \end{theorem}

  \begin{theorem}[Normalization]
    Given $\Gamma$, $\Tm{M}$, $\Tm{N}$ and $A$, it is decidable whether
    $\EqTm{M}{N}{A}$.
  \end{theorem}
  
  \textbf{These results give MLTT its unusual computational flavour.}
\end{frame}


\section{Examples}

\begin{frame}
  \frametitle{A familiar construction (I)}
  
  Let $\IsTy{A, B}[]$, and $\IsTy[x : A, y : B]{R(\Tm{x}, \Tm{y})}[]$. Then
  \[
    \inferrule{
      \inferrule*{
        \inferrule*{
          %
        }{
          \IsTy[x : A, y : B]{R(\Tm{x}, \Tm{y})}[]
        }
      }{
        \IsTy[x : A]{\DSum{y}{B}{R(\Tm{x}, \Tm{y})}}[]
      }
    }{
      \IsTy[]{\Fn{x}{A}{\DSum{y}{B}{R(\Tm{x}, \Tm{y})}}}[]
    }
  \]
  This is essentially $\forall x : A.\ \exists y : B.\ R(\Tm{x}, \Tm{y})$.
  
  \medskip
  
  Similarly, recalling that $A \to B \defeq \Fn{x}{A}{B}$, we have
  \[
    \inferrule{
      \inferrule*{\vdots}{
        \IsTy[f : A \to B, x : A]{R(\Tm{x}, \Tm{f(x)})}[]
      }
    }{
      \IsTy[]{\DSum{f}{A \to B}{\parens{\Fn{x}{A}{R(\Tm{x}, \Tm{f(x)})}}}}[] 
    }
  \]
  This is essentially $\exists f : A \to B.\ \forall x : A.\ R(\Tm{x}, \Tm{f(x)})$.
\end{frame}

\begin{frame}
  \frametitle{A familiar construction (II)}
  
  Let $\IsTy{A, B}[]$, and $\IsTy[x : A, y : B]{R(\Tm{x}, \Tm{y})}[]$. Then
  \[
    \IsTy[]{\Fn{x}{A}{\DSum{y}{B}{R(\Tm{x}, \Tm{y})}}}[]
  \]
  This is essentially $\forall x : A.\ \exists y : B.\ R(x, y)$.
  
  Similarly, recalling that $A \to B \defeq \Fn{x}{A}{B}$, we have
  \[
    \IsTy[]{\DSum{f}{A \to B}{\parens{\Fn{x}{A}{R(\Tm{x}, \Tm{f(x)})}}}}[] 
  \]
  This is essentially $\exists f : A \to B.\ \forall x : A.\ R(\Tm{x}, \Tm{f(x)})$.
  
  \begin{multline*}
    \IsTm{?}{
      \parens{\Fn{x}{A}{\DSum{y}{B}{R(\Tm{x}, \Tm{y})}}} \\ 
      \to \parens{ \DSum{f}{A \to B}{\parens{\Fn{x}{A}{R(\Tm{x}, \Tm{f(x)})}}} }
    } 
  \end{multline*}
  \pause
  Indeed, this is the \textbf{type-theoretic ``axiom'' of choice}:
  \begin{multline*}
    \IsTm{
      \Lam{g}{
        \Pair{
          \Lam{x}{\Proj[1]{\App{g}{x}}}
        }{
          \Lam{x}{\Proj[2]{\App{g}{x}}}
        }
      }
    }{
      \parens{\Fn{x}{A}{\DSum{y}{B}{R(\Tm{x}, \Tm{y})}}} \\ 
      \to \parens{ \DSum{f}{A \to B}{\parens{\Fn{x}{A}{R(\Tm{x}, \Tm{f(x)})}}} }
    } 
  \end{multline*}
\end{frame}

\begin{frame}
  \frametitle{The type-theoretic ``axiom'' of choice}
  
  Let $\IsTy{A, B}[]$, and $\IsTy[x : A, y : B]{R(\Tm{x}, \Tm{y})}[]$. Then
  \begin{multline*}
    \IsTm{
      \Lam{g}{
        \Pair{
          \Lam{x}{\Proj[1]{\App{g}{x}}}
        }{
          \Lam{x}{\Proj[2]{\App{g}{x}}}
        }
      }
    }{
      \parens{\Fn{x}{A}{\DSum{y}{B}{R(\Tm{x}, \Tm{y})}}} \\ 
      \to \parens{ \DSum{f}{A \to B}{\parens{\Fn{x}{A}{R(\Tm{x}, \Tm{f(x)})}}} }
    } 
  \end{multline*}
  Suppose $\Tm{g} : \Fn{x}{A}{\DSum{y}{B}{R(\Tm{x}, \Tm{y})}}$. Then clearly
  \begin{align*}
    \Tm{f} &\defeq \Tm{
      \Lam{x}[A]{
        \underbrace{\Proj[1]{\overbrace{\App{g}{x}}^{\DSum{y}{B}{R(\Tm{x}, \Tm{y})}}}}_{B}
      }
    } : A \to B 
    \\
    \Tm{h} &\defeq \Tm{
      \Lam{x}[A]{
        \underbrace{\Proj[2]{\overbrace{\App{g}{x}}^{\DSum{y}{B}{R(\Tm{x}, \Tm{y})}}}}_{
          R(\Tm{x}, \Proj[1]{\App{g}{x}})
        }
      } 
    } : \Fn{x}{A}{R(\Tm{x}, \Tm{\Proj[1]{\App{g}{x}}})}
  \end{align*}
  But $\Tm{\App{f}{x}} \equiv \Tm{\Proj[1]{\App{g}{x}}}$, so this type is equal
  to $\Fn{x}{A}{R(\Tm{x}, \Tm{\App{f}{x}})}$. Hence $\Tm{\Lam{g}{\Pair{f}{h}}}$ has the
  right type.
  
\end{frame}


\end{document}