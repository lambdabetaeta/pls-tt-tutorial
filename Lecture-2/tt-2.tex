\documentclass[handout]{beamer} % use [handout] option to stop pauses
\beamertemplatenavigationsymbolsempty 

\usepackage[utf8]{inputenc}
\usepackage{svg}
\usepackage{amsmath}
\usepackage{amssymb}
\usepackage{mathtools}
\usepackage{stmaryrd}
\usepackage{libertine}
\usepackage{xparse}
\usepackage{mathpartir}
\usepackage{fancybox}
\usepackage{../jmsdelim}
\usepackage{../macros}

\usecolortheme[named=BristolURed]{structure}

\AtBeginSection[]{
  \begin{frame}
  \vfill
  \centering
  \begin{beamercolorbox}[sep=12pt,center,shadow=true]{title}
		\usebeamerfont{title}
		\scshape
		\uppercase\expandafter{\romannumeral\insertsectionnumber\relax}.~\insertsectionhead\par%
  \end{beamercolorbox}
  \vfill
  \end{frame}
}

\newcommand\doubleplus{+\kern-1.3ex+\kern0.8ex}

\title{Type Theory and Homotopy \\ II. Equality, Universes, and Homotopy}
\author{
		Alex Kavvos % \inst{1} % \orcidID{0000-0001-7953-7975}
}

% \institute{
% 		% \inst{1}
%     University of Bristol
% 		% \email{alex.kavvos@bristol.ac.uk}
% }
\titlegraphic{\includegraphics[scale=0.1]{../Bristol.png}}
\date{Panhellenic Logic Symposium, 6--10 July 2022}


\begin{document}

\frame{\titlepage}

\begin{frame}
  \frametitle{Equality}

  Recall that we could define
  \[
    \IsTm[]{\textsf{add}\ {\color{black}{=}}\
      \Lam{x}{\Lam{y}{
        \NatRec{x}{y}{n}{c}{\Succ{n}}
      }}}{\Nat \to \Nat \to \Nat}
  \]
  and compute that
  \[
    \IsTm[y : \Nat]{
      \App{\App{\textsf{add}}{\Zero}}{y}
        \deq
      y
    }{\Nat}
  \]
  It is \textbf{not} the case that
  \[
    \IsTm[x : \Nat]{
      \App{\App{\textsf{add}}{x}}{\Zero}
        \deq
      x
    }{\Nat}
  \]
  $\deq$ only allows unfolding of definitions, \textbf{not} non-trivial
  theorems.
  
  \medskip
  
  For that we need to introduce the \textbf{identity type}.
\end{frame}



\section{Identity Types}



\begin{frame}
  \frametitle{Intensional Identity Types}
  \small
  
  \begin{center}
      \renewcommand{\arraystretch}{2.5}
    \begin{tabular}{p{1cm}c}
      form. &
      $
        \inferrule{
          \IsTm{M}{A} \\
          \IsTm{N}{A}
        }{
          \IsTy{\Id{A}{M}{N}}[]
        }
      $ \\
      intro. &
      $
        \inferrule{
          \IsTm{M}{A}
        }{
          \IsTm{\Refl{M}}{\Id{A}{M}{M}}
        }
      $ \\[2ex]
      elim. &
      $
        \inferrule{
          \IsTy[\Gamma, x : A, y : A, p : \Id{A}{x}{y}]{B}[] \\
          \IsTm{P}{\Id{A}{M}{N}} \\
          \IsTm[\Gamma, z : A]{Q}{\Sb{B}{z, z, \Refl{z}}{x, y, p}}
        }{
          \IsTm{
            \IdRec[x, y, p. B]{P}{z}{Q}
          }{\Sb{B}{M, N, P}{x, y, p}}
        }
      $ 
      \\[2ex]
      comp. &
      $
        \inferrule{
          \IsTy[\Gamma, x : A, y : A, p : \Id{A}{x}{y}]{B}[] \\
          \IsTm[\Gamma, z : A]{Q}{\Sb{B}{z, z, \Refl{z}}{x, y, p}}
        }{
          \EqTm{
            \IdRec{\Refl{M}}{z}{Q}
          }{
            \Sb{Q}{M}{z}
          }{
            \Sb{B}{M, M, \Refl{M}}{x, y, p}
          }
        }
      $
    \end{tabular}
  \end{center}
  Because of the \textbf{type conversion} and \textbf{congruence} rules we
  always have
  \begin{mathpar}
    \inferrule{
      \EqTm{M}{N}{A}
    }{
      \IsTm{\Refl{M}}{\Id{A}{M}{N}}
    }
  \end{mathpar}
\end{frame}

\begin{frame}
  \frametitle{Some examples (I)}
  \small
  
  \begin{itemize}
    \item Let $\IsTy[]{A}[]$ and $\IsTy[x : A]{P(x)}[]$. We have:
     \[ 
       \IsTm[x, y : A, p : \Id{A}{x}{y}]{
        \Transport{p} \deq \IdRec{p}{z}{\Lam{w}{w}}
       }{B(x) \to B(y)}
    \]
    Informally:
    \begin{quote}
      \normalfont
      Let $x, y : A$ and $p : \Id{A}{x}{y}$. We want to construct a term of type
      $B(x) \to B(y)$. By \textbf{induction} we may assume that $x \equiv y$, so
      it suffices to give a term $B(x) \to B(x)$. Take the identity function.
    \end{quote}

    \item Let $\IsTm[x : A]{f(x)}{B}$. Then $\IsTy[x, y :
      A]{\Id{B}{f(x)}{f(y)}}[]$. 
    
    We have
      \[
        \IsTm[x, y : A, p : \Id{A}{x}{y}]{
          \Ap{f}{p} \deq \IdRec{p}{x}{\Refl{f(x)}}
        }{\Id{B}{f(x)}{f(y)}}
      \]
      Informally:
      \begin{quote}
        \normalfont
        Let $x, y : A$ and $p : \Id{A}{x}{y}$. We want to show
        $\Id{B}{f(x)}{f(y)}$. By \textbf{induction} we may assume that $x \equiv
        y$, so it suffices to construct a term of type $\Id{B}{f(x)}{f(x)}$.
        Take $\Tm{\Refl{f(x)}}$.
      \end{quote}
  \end{itemize}
  
\end{frame}


\begin{frame}
  \frametitle{Some examples (II)}
  
  Here is an informal proof that there is a term of type
  \[
    \IsTy[x : \Nat]{
      \Id{\Nat}{
        \App{\App{\textsf{add}}{x}}{\Zero}
      }{
        x
      }
    }[]
  \]
  We proceed by induction on $x : \Nat$.
  \begin{itemize}
    \item If $x \deq \Tm{\Zero} : \Nat$, then 
      $
      \Tm{\App{\App{\textsf{add}}{x}}{\Zero}}
        \deq
      \Tm{\App{\App{\textsf{add}}{\Zero}}{\Zero}}
        \deq 
      \Tm{\Zero}
      $.
      Hence it suffices to construct $\Tm{\Refl{\Zero}} : \Id{\Nat}{\Zero}{\Zero}$.
    \item If $x \deq \Tm{\Succ{y}} : \Nat$ for some $y : \Nat$, then
    \[
      \Tm{\App{\App{\textsf{add}}{x}}{\Zero}}
        \deq
      \Tm{\App{\App{\textsf{add}}{\Succ{y}}}{\Zero}}
        \deq
      \Tm{\Succ{\App{\App{\textsf{add}}{y}}{\Zero}}}
    \]
    By the IH we have $p : \Id{\Nat}{\App{\App{\textsf{add}}{y}}{\Zero}}{y}$.
    Hence 
    \[
    \Tm{\Ap{\Succ{-}}{p}} : \Id{\Nat}{
      \underbrace{\Succ{\App{\App{\textsf{add}}{y}}{\Zero}}}_{
        \deq\ \App{\App{\textsf{add}}{x}}{\Zero}
      }
    }{
      \underbrace{\Succ{y}}_{\deq\ x}
    }
    \]

    So we have shown the inductive step.
  \end{itemize}
    

\end{frame}


\begin{frame}
  \frametitle{Metatheory}

  \begin{theorem}
    The following rule is admissible.
    \begin{mathpar}
      \inferrule{
        \IsTm[]{P}{\Id{A}{M}{N}}
      }{
        \EqTm[]{M}{N}{A}
      }
    \end{mathpar}
  \end{theorem}

  Any two propositionally equal terms in an \textbf{empty context}\\ are also definitionally equal.
  
  \medskip
  
  This did not apply to our previous proof because $x : \Nat$ was free.
  
  \medskip

  Hence the name `intensional.'
\end{frame}

\begin{frame}
  \frametitle{Extensional Identity Types}
  
  One might argue that $
    \IsTm[x : \Nat]{\ldots}{
      \Id{\Nat}{
        \App{\App{\textsf{add}}{x}}{\Zero}
      }{
        x
      }
    }
  $
  should be promoted to a definitional equality
  \[
    \EqTm[x : \Nat]{
      \App{\App{\textsf{add}}{x}}{\Zero}
    }{
      x
    }{\Nat}
  \]
  Add \textbf{equality reflection} rule: % and \textbf{equality uniqueness} rules:
  \begin{mathpar}
    \inferrule{
      \IsTm{P}{\Id{A}{M}{N}}
    }{
      \EqTm{M}{N}{A}
    }
    % \and
    % \inferrule{
    %   \IsTm{P}{\Id{A}{M}{N}}
    % }{
    %   \EqTm{P}{\Refl{M}}{\Id{A}{M}{M}}
    % }
  \end{mathpar}
  We then say we have \textbf{extensional identity types}. But then
  \begin{itemize}
    \item normalization is no longer decidable, and hence
    \item \textbf{type checking is no longer decidable}
  \end{itemize}
  
  % I asked Daniel whether we need the uniqueness rule, and our consensus
  % seems to be that J + computation + eq refl <=> eq refl + eq uniq.

  So we are stuck with the `bureaucracy' of intensional identity types.
  
  \medskip
  
  But this is a fine type theory for computing by hand.

  (It is the \textbf{internal language} of locally cartesian closed categories.)

\end{frame}

\section{Universes}

\begin{frame}
  \frametitle{Identity types are not good enough}
  
  \begin{theorem}[Jan Smith, J. Symb. Log. 1988]
    The type correspondong to Peano's fourth axiom, i.e.
    \[
      \IsTy[n : \Nat]{\Id{\Nat}{0}{\Succ{n}} \to \mathbf{0}}[]
    \]
    is \textbf{not} inhabited in MLTT with $\Pi$, $\Sigma$, and identity types.
  \end{theorem}
  Proof: construct a model of MLTT where types are subsingleton sets.

  \medskip

  To prove Peano 4, we intuitively want to
  \begin{enumerate}
    \item \Alert{construct a type family} $\IsTy[n : \Nat]{B(n)}[]$ for which
      $B(\Tm{\Zero})$ is inhabited, while $\IsTy[n : \Nat]{B(\Tm{\Succ{n}})}[]$
      is always empty.
    \item assuming $\IsTm[n : \Nat]{p}{\Id{\Nat}{0}{\Succ{n}}}$ and
      $\IsTm[]{M}{B(\Tm{\Zero})}$, obtain $\IsTm[n :
      \Nat]{\App{\Transport{p}}{M}}{B(\Tm{\Succ{n}})} \deq \textbf{0}$
  \end{enumerate}
  We cannot perform Step 1 because types are not terms.
\end{frame}

\begin{frame}
  \frametitle{Universes \`{a} la Russell}
  We introduce the \textbf{universe}, a \textbf{type of all (small) types}.
  \begin{mathpar}
    \inferrule{ }{\IsTy{\Uni}[]}
    \and
    \inferrule{
      \IsTm{A}{\Uni}
    }{
      \IsTy{A}[]
    }
  \end{mathpar}
  plus one rule for each type constructor, e.g.
  \begin{mathpar}
    \inferrule{
      \IsTm{A}{\Uni} \\
      \IsTm[\Gamma, x : \Uni]{B}{\Uni}
    }{
      \IsTm{\Fn{x}{A}{B}}{\Uni}
    }
  \end{mathpar}
  \Alert{Caution}. We must \textbf{not} have
  \[
    \inferrule{ }{\IsTm{\Uni}{\Uni}}
  \]
  because that causes a Russell-style paradox (Girard's paradox).
  
  \medskip
  
  Types may then be constructed as terms of $\Uni$ (e.g. by induction).
\end{frame}


\section{Homotopy}

\begin{frame}
  \frametitle{Intensional identity types very mysterious}
  
  why so difficult

  extensional bad because undecidable

  so why mysterious?
  
  homotopy!
\end{frame}

\begin{frame}
  \frametitle{Models of intensional identity types}
  
  Hofmann Streicher model

  UIP principle
\end{frame}

\begin{frame}
  \frametitle{Homotopy theory}

  homotopi
\end{frame}

\begin{frame}
  \frametitle{Homotopy \emph{type} theory}
  
  Awodey/Warren and Voevodsky
\end{frame}

\begin{frame}
  \frametitle{Extensionality principles}
  
\end{frame}

\begin{frame}
  \frametitle{Univalence}
  
\end{frame}

\begin{frame}
  \frametitle{References}
  \bibliographystyle{amsalpha}
  \bibliography{../refs.bib}
  \nocite{hott_2013}
  \nocite{awodey_2012}
  \nocite{shulman_2017}
\end{frame}


\end{document}