\documentclass[handout]{beamer} % use [handout] option to stop pauses
\beamertemplatenavigationsymbolsempty 

\usepackage[utf8]{inputenc}
\usepackage{svg}
\usepackage{amsmath}
\usepackage{amssymb}
\usepackage{mathtools}
\usepackage{stmaryrd}
\usepackage{libertine}
\usepackage{xparse}
\usepackage{mathpartir}
\usepackage{fancybox}
\usepackage{graphicx}
\usepackage{tikz}
\usepackage{../jmsdelim}
\usepackage{../macros}
\usepackage{../categories}

\usecolortheme[named=BristolURed]{structure}

\AtBeginSection[]{
  \begin{frame}
  \vfill
  \centering
  \begin{beamercolorbox}[sep=12pt,center,shadow=true]{title}
		\usebeamerfont{title}
		\scshape
		\uppercase\expandafter{\romannumeral\insertsectionnumber\relax}.~\insertsectionhead\par%
  \end{beamercolorbox}
  \vfill
  \end{frame}
}

\newcommand\doubleplus{+\kern-1.3ex+\kern0.8ex}

\title{Type Theory and Homotopy \\ II. Identity and Homotopy}
\author{
		Alex Kavvos % \inst{1} % \orcidID{0000-0001-7953-7975}
}

% \institute{
% 		% \inst{1}
%     University of Bristol
% 		% \email{alex.kavvos@bristol.ac.uk}
% }
\titlegraphic{\includegraphics[scale=0.1]{../Bristol.png}}
\date{Panhellenic Logic Symposium, 6--10 July 2022}


\begin{document}

\frame{\titlepage}

\begin{frame}
  \frametitle{Equality}

  Recall that we could define
  \[
    \IsTm[]{\textsf{add}\ {\color{black}{=}}\
      \Lam{x}{\Lam{y}{
        \NatRec{x}{y}{n}{c}{\Succ{n}}
      }}}{\Nat \to \Nat \to \Nat}
  \]
  and compute that
  \[
    \IsTm[y : \Nat]{
      \App{\App{\textsf{add}}{\Zero}}{y}
        \deq
      y
    }{\Nat}
  \]
  It is \textbf{not} the case that
  \[
    \IsTm[x : \Nat]{
      \App{\App{\textsf{add}}{x}}{\Zero}
        \deq
      x
    }{\Nat}
  \]
  $\deq$ only allows unfolding of definitions, \textbf{not} non-trivial
  theorems.
  
  \medskip
  
  For that we need to introduce the \textbf{identity type}.
\end{frame}



\section{Identity Types}



\begin{frame}
  \frametitle{Intensional Identity Types}
  \small
  
  \begin{center}
      \renewcommand{\arraystretch}{2.5}
    \begin{tabular}{p{1cm}c}
      form. &
      $
        \inferrule{
          \IsTm{M}{A} \\
          \IsTm{N}{A}
        }{
          \IsTy{\Id{A}{M}{N}}[]
        }
      $ \\
      intro. &
      $
        \inferrule{
          \IsTm{M}{A}
        }{
          \IsTm{\Refl{M}}{\Id{A}{M}{M}}
        }
      $ \\[2ex]
      elim. &
      $
        \inferrule{
          \IsTy[\Gamma, x : A, y : A, p : \Id{A}{x}{y}]{B}[] \\
          \IsTm{P}{\Id{A}{M}{N}} \\
          \IsTm[\Gamma, z : A]{Q}{\Sb{B}{z, z, \Refl{z}}{x, y, p}}
        }{
          \IsTm{
            \IdRec[x, y, p. B]{P}{z}{Q}
          }{\Sb{B}{M, N, P}{x, y, p}}
        }
      $ 
      \\[2ex]
      comp. &
      $
        \inferrule{
          \IsTy[\Gamma, x : A, y : A, p : \Id{A}{x}{y}]{B}[] \\
          \IsTm[\Gamma, z : A]{Q}{\Sb{B}{z, z, \Refl{z}}{x, y, p}}
        }{
          \EqTm{
            \IdRec{\Refl{M}}{z}{Q}
          }{
            \Sb{Q}{M}{z}
          }{
            \Sb{B}{M, M, \Refl{M}}{x, y, p}
          }
        }
      $
    \end{tabular}
  \end{center}
  Because of the \textbf{type conversion} and \textbf{congruence} rules we
  always have
  \begin{mathpar}
    \inferrule{
      \EqTm{M}{N}{A}
    }{
      \IsTm{\Refl{M}}{\Id{A}{M}{N}}
    }
  \end{mathpar}
\end{frame}

\begin{frame}
  \frametitle{Some examples (I)}
  \small
  
  \begin{itemize}
    \item Let $\IsTy[]{A}[]$ and $\IsTy[x : A]{P(x)}[]$. We have:
     \[ 
       \IsTm[x, y : A, p : \Id{A}{x}{y}]{
        \Transport{p} \deq \IdRec{p}{z}{\Lam{w}{w}}
       }{B(x) \to B(y)}
    \]
    Informally:
    \begin{quote}
      \normalfont
      Let $x, y : A$ and $p : \Id{A}{x}{y}$. We want to construct a term of type
      $B(x) \to B(y)$. By \textbf{induction} we may assume that $x \equiv y$, so
      it suffices to give a term $B(x) \to B(x)$. Take the identity function.
    \end{quote}

    \item Let $\IsTm[x : A]{f(x)}{B}$. Then $\IsTy[x, y :
      A]{\Id{B}{f(x)}{f(y)}}[]$. 
    
    We have
      \[
        \IsTm[x, y : A, p : \Id{A}{x}{y}]{
          \Ap{f}{p} \deq \IdRec{p}{x}{\Refl{f(x)}}
        }{\Id{B}{f(x)}{f(y)}}
      \]
      Informally:
      \begin{quote}
        \normalfont
        Let $x, y : A$ and $p : \Id{A}{x}{y}$. We want to show
        $\Id{B}{f(x)}{f(y)}$. By \textbf{induction} we may assume that $x \equiv
        y$, so it suffices to construct a term of type $\Id{B}{f(x)}{f(x)}$.
        Take $\Tm{\Refl{f(x)}}$.
      \end{quote}
  \end{itemize}
  
\end{frame}


\begin{frame}
  \frametitle{Some examples (II)}
  
  Here is an informal proof that there is a term of type
  \[
    \IsTy[x : \Nat]{
      \Id{\Nat}{
        \App{\App{\textsf{add}}{x}}{\Zero}
      }{
        x
      }
    }[]
  \]
  We proceed by induction on $x : \Nat$.
  \begin{itemize}
    \item If $x \deq \Tm{\Zero} : \Nat$, then 
      $
      \Tm{\App{\App{\textsf{add}}{x}}{\Zero}}
        \deq
      \Tm{\App{\App{\textsf{add}}{\Zero}}{\Zero}}
        \deq 
      \Tm{\Zero}
      $.
      Hence it suffices to construct $\Tm{\Refl{\Zero}} : \Id{\Nat}{\Zero}{\Zero}$.
    \item If $x \deq \Tm{\Succ{y}} : \Nat$ for some $y : \Nat$, then
    \[
      \Tm{\App{\App{\textsf{add}}{x}}{\Zero}}
        \deq
      \Tm{\App{\App{\textsf{add}}{\Succ{y}}}{\Zero}}
        \deq
      \Tm{\Succ{\App{\App{\textsf{add}}{y}}{\Zero}}}
    \]
    By the IH we have $p : \Id{\Nat}{\App{\App{\textsf{add}}{y}}{\Zero}}{y}$.
    Hence 
    \[
    \Tm{\Ap{\Succ{-}}{p}} : \Id{\Nat}{
      \underbrace{\Succ{\App{\App{\textsf{add}}{y}}{\Zero}}}_{
        \deq\ \App{\App{\textsf{add}}{x}}{\Zero}
      }
    }{
      \underbrace{\Succ{y}}_{\deq\ x}
    }
    \]

    So we have shown the inductive step.
  \end{itemize}
    

\end{frame}


\begin{frame}
  \frametitle{Metatheory}

  \begin{theorem}
    The following rule is admissible.
    \begin{mathpar}
      \inferrule{
        \IsTm[]{P}{\Id{A}{M}{N}}
      }{
        \EqTm[]{M}{N}{A}
      }
    \end{mathpar}
  \end{theorem}

  Any two propositionally equal terms in an \textbf{empty context}\\ are also definitionally equal. (Hence the name `intensional.')
  
  \medskip
  
  This did not apply to our previous proof because $x : \Nat$ was free.
  
  \medskip

  \begin{theorem}
    There is a set-theoretic model of MLTT with $\Pi$, $\Sigma$, $\textsf{Id}$, $\Nat$, and $+$ types.
  \end{theorem}
\end{frame}

\begin{frame}
  \frametitle{Extensional Identity Types}
  
  One might argue that $
    \IsTm[x : \Nat]{\ldots}{
      \Id{\Nat}{
        \App{\App{\textsf{add}}{x}}{\Zero}
      }{
        x
      }
    }
  $
  should be promoted to a definitional equality
  \[
    \EqTm[x : \Nat]{
      \App{\App{\textsf{add}}{x}}{\Zero}
    }{
      x
    }{\Nat}
  \]
  Add \textbf{equality reflection} rule: % and \textbf{equality uniqueness} rules:
  \begin{mathpar}
    \inferrule{
      \IsTm{P}{\Id{A}{M}{N}}
    }{
      \EqTm{M}{N}{A}
    }
    % \and
    % \inferrule{
    %   \IsTm{P}{\Id{A}{M}{N}}
    % }{
    %   \EqTm{P}{\Refl{M}}{\Id{A}{M}{M}}
    % }
  \end{mathpar}
  We then say we have \textbf{extensional identity types}. But then
  \begin{itemize}
    \item normalization is no longer decidable, and hence
    \item \textbf{type checking is no longer decidable}
  \end{itemize}
  
  % I asked Daniel whether we need the uniqueness rule, and our consensus
  % seems to be that J + computation + eq refl <=> eq refl + eq uniq.

  So we are stuck with the `bureaucracy' of intensional identity types.
  
  \medskip
  
  But this is a fine type theory for computing by hand.

  % (It is the \textbf{internal language} of locally cartesian closed categories.)

\end{frame}


\section{Homotopy}

\begin{frame}
  \frametitle{Identity types are very mysterious}
  Let $\IsTm[]{M, N}{A}$. Construct $\IsTy[]{\Id{A}{M}{N}}[]$.

  \medskip

  Now suppose $\IsTm[]{P, Q}{\Id{A}{M}{N}}$.

  \medskip

  What is the meaning of the following type?
  \[
    \IsTy[]{\Id{\Id{A}{M}{N}}{P}{Q}}[]
  \]

  Should the following be inhabited for any type $\IsTy{A}[]$?
  \[
    \IsTy[]{
      \Fn{x, y}{A}{
        \Fn{p, q}{\Id{A}{x}{y}}{
          \Id{\Id{A}{x}{y}}{p}{q}
        }
      }
    }[]
  \]
  It's certainly true in the set-theoretic model!
  \begin{theorem}[Hofmann-Streicher, 1998]
    There is a model of MLTT in which the above principle of \textbf{uniqueness of identity proofs} (UIP) is \textbf{not} true.
  \end{theorem}
\end{frame}

\begin{frame}
  \frametitle{Groupoids}
  
  \begin{definition}
    A \textbf{groupoid} $\mathcal{G}$ consists of
    \begin{itemize}
      \item a set of \textbf{objects} $\Ob{\mathcal{G}}$
      \item for $x, y \in \Ob{\mathcal{G}}$ a set of \textbf{isomorphisms} $\Hom{x}{y}$
        
      We write $f : x \Iso y$ if $f \in \Hom{x}{y}$.
      \item for each $x \in \Ob{\mathcal{G}}$ an \textbf{identity} $1_x \in \Hom{x}{x}$
      \item for isos $f : x \Iso y$ and $g : y \Iso z$ a \textbf{composite}
      \[
        g \circ f : x \Iso z
      \]
      \item for each iso $f : x \Iso y$ and \textbf{inverse iso} $f^{-1} : y \Iso x$,
      such that
      \begin{align*}
        f^{-1} \circ f &= 1_x : x \Iso x &
        f \circ f^{-1} &= 1_y : y \Iso y
      \end{align*}
    \end{itemize}
  \end{definition}
  
  A \textbf{one-object} groupoid is \dots \pause a group!

  \smallskip
  
  If $\lvert \Hom{x}{y} \rvert \leq 1$ a groupoid is \dots  \pause an equivalence
  relation!
\end{frame}

\begin{frame}
  \frametitle{The Hofmann-Streicher groupoid model of type theory}
  
  Hofmann and Streicher interpreted MLTT as follows:
  \begin{itemize}
    \item $\IsTy[]{A}[]$ is interpreted by a groupoid $\Interp{A}$.
    \item A type family/dependent type $\IsTy[x : A]{B}[]$ is interpreted by a
      \textbf{fibration} $\Interp{B} : \Interp{A} \to \GPD$ of groupoids.
    % In particular, if $p : x \Iso y$ in $\Interp{A}$ then we have an isomorphism of groupoids $\Interp{B}(p) : \Interp{B}(x) \Iso \Interp{B}(y)$.
    \item A term of type $\IsTy[x : A]{B}[]$ is a \textbf{section} of the fibration $\Interp{B}$.
    \item The identity type $\IsTy[]{\Id{A}{M}{N}}[]$ is interpreted by the set
      of isomorphisms of the groupoid $\Interp{A}$, i.e.
    \[
      \Hom[\Interp{A}]{\Interp{M}}{\Interp{N}}
    \]
  \end{itemize}

  \medskip

  In this model there are types with \textbf{non-trivial identity types}.

  \medskip
  
  But where do groupoids come from?
\end{frame}

\begin{frame}
  \frametitle{Paths}
  
  Let $X$ be a (topological) space.

  \begin{definition}
    A \textbf{path} in space $X$ is a continuous function $p : [0, 1] \to X$.

  \end{definition}

  \medskip
  
  Write $p : x \Path y$ if $p(0) = x$ and $p(1) = y$.
  
  \medskip
  
  Given $p : x \Path y$ let $p^{-1} : y \Path x$ by $p^{-1}(t) \defeq p(1 - t)$.

  \medskip

  Given $p : x \Path y$ and $q : y \Path z$ let
  \[
    (p \Concat q)(t) \defeq
      \begin{cases}
        p(2t) & \text{ if $0 \leq t \leq 1/2$} \\
        q(2t - 1) & \text{ if $1/2 \leq t \leq 1$} \\
      \end{cases}
  \]
  
  \textbf{Question}: given 
  \begin{align*}
    p &: x \Path y
    &
    q &: y \Path z
    &
    r &: z \Path w
  \end{align*}
  is the following true?
  \[
    (p \Concat q) \Concat r \overset{?}{=} p \Concat (q \Concat r)
  \]
\end{frame}

\begin{frame}
  \frametitle{Homotopy}

  Let $f, g : X \to Y$ be continuous functions.

  \begin{definition}
    A \textbf{homotopy} $H$ from $f$ to $g$ is a continuous function
    \[
      H : X \times [0, 1] \to Y
    \]
    such that $H(-, 0) = f$ and $H(-, 1) = g$.
  \end{definition}
  
  \begin{center}
    \includegraphics[scale=0.2]{Homotopy.jpg}
  \end{center}

  Write $f \simeq g$ if there is a homotopy from $f$ to $g$.

  $\simeq$ is an equivalence relation.
\end{frame}

% \begin{frame}
%   \frametitle{Homotopy (II)}
%   \begin{theorem}
%     $\simeq$ is an equivalence relation.
%   \end{theorem}
  % \begin{proof}
  %   \small 
  %   \textbf{Transitivity}. Let $f, g, h : X \to Y$, with $f \simeq g$ and $g \simeq h$.
  %   Given $H_1 : X \times [0, 1] \to Y$ from $f$ to $g$ and $H_2 : X \times [0,
  %   1] \to Y$ from $g$ to $h$, define 
  %   \[
  %     H_3(x, t) \defeq \begin{cases}
  %       H_1(x, t) & \text{ when } 0 \leq t < 1/2 \\
  %       H_2(x, t) & \text{ when } 1/2 \leq t \leq 1
  %     \end{cases}
  %   \]
  %   Hence $f \simeq h$.

  %   \textbf{Symmetry}. Let $f \simeq g$. Given $H : X \times [0, 1] \to Y$
  %   define
  %   \[
  %     H'(x, t) \defeq H(x, 1 - t)
  %   \]
  %   Hence $g \simeq f$.
  % \end{proof}
% \end{frame}

\begin{frame}
  \frametitle{Associativity and Homotopy}
  \small
  Given 
  \begin{align*}
    p &: x \Path y
    &
    q &: y \Path z
    &
    r &: z \Path w
  \end{align*}
  we have that
  \[
    (p \Concat q) \Concat r \simeq p \Concat (q \Concat r) : x \Path w
  \]

  \begin{center}
    \includegraphics[scale=0.3]{assoc.jpg}
  \end{center}
  
  If $1_x : x \Path x$ and $1_y : y \Path y$ are constant paths then
  $
    p \Concat 1_y \simeq p \simeq 1_x \Concat p
  $.
\end{frame}

\begin{frame}
  \frametitle{The Fundamental Groupoid}

  Let $X$ be a space. Its \textbf{fundamental groupoid} $\pi(X)$ consists of
  \begin{description}
    \item[objects] the points of $X$
    \item[isomorphisms] equiv. classes $[p]$ of paths $p : x \Path y$ up to
      $\simeq$
  \end{description}
  
  \medskip
  
  Taking only equivalence classes of \textbf{loops} $p : x \Path x$ at $x \in X$
  gives the \textbf{fundamental group} $\pi(X, x)$ of $X$ at $x$.
  
  \medskip
  
  These are essential \textbf{algebraic invariants} of the space $X$.
  
  \[
    \mathbb{S}^1 \quad\defeq\quad
    \begin{tikzpicture}[baseline=(current  bounding  box.center)]
      \filldraw[color=BristolURed, fill=none, very thick](0,0) circle (1.2);
      \filldraw (1.2,0) circle (2pt) node[right] {$b$};
    \end{tikzpicture}
  \]

  \begin{theorem}
    $\pi(\mathbb{S}^1, b) \cong \mathbb{Z}$
  \end{theorem}
\end{frame}

\begin{frame}
  \frametitle{$\infty$-Groupoids}
  
  The fundamental insight: 
  \begin{center}
    \shadowbox{Why quotient at all?}
  \end{center}

  \begin{definition}
    A \textbf{groupoid} $\mathcal{G}$ consists of
    \begin{itemize}
      \item a set of \textbf{objects} $\Ob{\mathcal{G}}$
      \item for $x, y \in \Ob{\mathcal{G}}$ a set of \textbf{isomorphisms} $\Hom{x}{y}$

      \vdots
    \end{itemize}
  \end{definition}

  \begin{definition}[sort of]
    An \textbf{$\infty$-groupoid} $\mathcal{G}$ consists of
    \begin{itemize}
      \item a set of \textbf{objects} $\Ob{\mathcal{G}}$
      \item for $x, y \in \Ob{\mathcal{G}}$ an \textbf{$\infty$-groupoid} $\Hom{x}{y}$

      \vdots
    \end{itemize}
  \end{definition}
\end{frame}

\begin{frame}
  \frametitle{The Fundamental $\infty$-Groupoid}

  Let $X$ be a space. Its \textbf{fundamental $\infty$-groupoid} $\pi_\infty(X)$
  consists of
  \begin{description}
    \item[0-cells] the points of $X$
    \item[1-cells] paths $p : x \Path y$ between points
    \item[2-cells] homotopies $H : p \simeq q$ between paths

    \vdots
  \end{description}
  
  \medskip

  Exact definition(s) tiresome to describe \textbf{analytically}.

  \medskip

  Grothendieck's (1928--2014) dream, aka the \textbf{homotopy hypothesis}:
  \begin{center}
    \shadowbox{$\infty$-groupoids = topological spaces up to homotopy}

    \includegraphics[scale=0.3]{mug.jpg}
  \end{center}
\end{frame}

\begin{frame}
  \frametitle{Identity Types and Homotopy}

  The intended allegory:
  \begin{center}
    \shadowbox{
      \begin{minipage}{0.80\textwidth}
        \centering
          types = spaces = $\infty$-groupoids

          elements of the identity type = paths in the space
      \end{minipage}
    }
  \end{center}
  
  For example, given $\IsTy[]{A}[]$ we can write down a term
  \[
    \Concat : \Fn{x, y, z}{A}{
      \Id{A}{x}{y}
      \to
      \Id{A}{y}{z}
      \to
      \Id{A}{x}{z}
    }
  \]
  \textbf{Informal proof}: Suppose $x, y, z : A$, $p : \Id{A}{x}{y}$, and $q :
  \Id{A}{y}{z}$. By the elimination rule we may assume that $x \deq y$ and $y
  \deq z$, so it suffices to define a term of type $\Id{A}{x}{x}$. Take
  $\Refl{x}$.
  
  \medskip
  
  Remember that because of the \textbf{computation rule} we have
  \[
    \Refl{x} \Concat \Refl{x} \deq \Refl{x}
  \]
\end{frame}

\begin{frame}
  \frametitle{Associativity of composition}
  
  \small
  
  Given $x, y, z : A$ we can then define a term
  \begin{multline*}
    \textsf{assoc}_{xyz} :
    \Fn{p}{\Id{A}{x}{y}}{
      \Fn{q}{\Id{A}{y}{z}}{
        \Fn{r}{\Id{A}{z}{w}}{ \\
          \Id{\Id{A}{x}{w}}{(p \Concat q) \Concat r}{p \Concat (q \Concat r)}
        }
      }
    }
  \end{multline*}
  
  \textbf{Informal proof}. Given $p, q, r$ as above we may assume that $x \deq y
  \deq z \deq w$ and $p \deq q \deq r \deq \Refl{x}$. Thus, need only a term of
  type
  \[
    \Id{\Id{A}{x}{x}}{
      \underbrace{(p \Concat q) \Concat r}_{
        \deq\ \Refl{x}
      }
    }{
      \underbrace{p \Concat (q \Concat r)}_{
        \deq\ \Refl{x}
      }
    }
  \]
  Take $\Refl{\Refl{x}}$.
  
  \medskip
  
  This can be taken to its logical conclusion:
  \begin{center}
    \shadowbox{
      \begin{minipage}{0.80\textwidth}
        \centering
          The elimination rule of the identity type \\
          generates the structure of an $\infty$-groupoid.
      \end{minipage}
    }
  \end{center}
  In other words, MLTT is a \textbf{synthetic} theory of $\infty$-groupoids.
\end{frame}

\begin{frame}
  \frametitle{Summary}
  \begin{itemize}
    \item \textbf{Intensional identity types} allow proofs of non-trivial, non-definitional equalities in MLTT.
    \item Iterated identity types are a very mysterious entity.
    \item They seem to naturally generate the structure of an $\infty$-groupoid.
    \item MLTT can be seen as a synthetic theory of $\infty$-groupoids.
  \end{itemize}
  
\end{frame}

\begin{frame}
  \frametitle{References}
  \bibliographystyle{amsalpha}
  \bibliography{../refs.bib}
  \nocite{hott_2013}
  \nocite{awodey_2012}
  \nocite{shulman_2017}
  \nocite{may_1999}
\end{frame}


\end{document}